%%%%%%%%%%%%%%%%%%%%%%%%%%%%%%%%%%%%%%%%%%%%%%%%%%%%%%%%%%%%%%%%%%%%%%%%%%%%%%%80
% preambulo.tex
% Preâmbulo do documento principal (tese.tex)
%
% Recomendo que você não use esse preâmbulo indiscriminadamente (mesmo que a
% compilação ocorra sem problemas), e que faça um esforço para ler (mesmo que
% pouco) a documentação de cada um dos pacotes utilizados. Neste caso, além dos
% já conhecidos www.google.com e http://tex.stackexchange.com/, também é
% recomendável dar uma olhada no wikibook do LaTeX:
%
% http://en.wikibooks.org/wiki/LaTeX
%
% Os pacotes usados aqui são os mais básicos, referentes à formatação de texto,
% legendas nas figuras e coisas do tipo. Se você tem necessidades mais
% específicas, você pode procurar no Google, adicionando os termos de busca
% 'LaTeX' e 'packages'. Um endereço muito útil é o http://ctan.org/, que é um
% repositório bem abrangente do TeX.
%%%%%%%%%%%%%%%%%%%%%%%%%%%%%%%%%%%%%%%%%%%%%%%%%%%%%%%%%%%%%%%%%%%%%%%%%%%%%%%80

%%%%%%%%%%%%%%%%%%%%%%%%%%%%%%%%%%%%%%%%%%%%%%%%%%%%%%%%%%%%%%%%%%%%%%%%%%%%%%%80
% Formato básico da página
%
% Para os ajustes, usei basicamente o wikibook:
%
% http://en.wikibooks.org/wiki/LaTeX/Page_Layout
%
% Se você quiser brincar um pouco com os ajustes das dimensões, uma dica: use o
% pacote showframe, descomentando a linha abaixo:
%\usepackage{showframe}
% Isso irá mostrar numa grade as margens do documento
\usepackage{geometry}
% Dimensões para impressão, de acordo com o manual da UFC.
\geometry{a4paper,inner=4cm, outer=2cm, top=3cm, bottom=2cm}

\pdfpagewidth=\paperwidth 
\pdfpageheight=\paperheight
% This acts as a failsafe to ensure things aren't stretched or moved when it's finally printed as a PDF.


\usepackage{ifxetex}
\ifxetex
    % XeLaTeX
    \usepackage{polyglossia}
    \usepackage{fontspec}
\else
    % default: pdfLaTeX
    \usepackage[brazil]{babel}
    \usepackage[T1]{fontenc}
    \usepackage[utf8]{inputenc}
    \usepackage[babel=true]{microtype}
\fi
